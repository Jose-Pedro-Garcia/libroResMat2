\subsection*{Sobre este material}

Este documento integra los materiales de apoyo para \textit{asistir} al estudiante en su proceso de aprendizaje de los contenidos de la Unidad Curricular \textit{Resistencia de Materiales 2} de la Carrera Ingeniería Civil brindada por la Facultad de Ingeniería de la Universidad de la República. %
%
El material no está editado y se encuentra en constante proceso de corrección y mejora. Se invita a estudiantes o lectores a colaborar publicando errores o contribuir realizando modificaciones a través del repositorio abierto \href{https://gitlab.fing.edu.uy/jorgepz/codigoFuenteLibroR2}{gitlab.fing.edu.uy/jorgepz/codigoFuenteLibroR2}. %
%
Este documento recibió contribuciones de los docentes ayudantes que han formado parte de la Unidad Curricular, particularmente en el planteo y resolución de los ejercicios prácticos. %
%
El docente Bruno Bazzano realizó aportes al contenido de la Unidad Temática 7, Diego Figueredo contribuyó con contenido de las Unidades Temáticas 6 y 7 y las soluciones, y Joaquín Viera ha contribuido con el apéndice de tablas. %
%
Se agradece a la docente Ximena Otegui, de la Unidad de Enseñanza de Facultad de Ingeniería (UEFI), por sus comentarios sobre el texto y la Unidad Curricular. %
%
Se agradece el apoyo financiero de la Comisión Sectorial de Enseñanza de la Universidad de la República, cuyo financiamiento del proyecto de innovaciones educativas titulado \textit{Rediseño de prácticas de enseñanza y evaluación en Resistencia de Materiales} permitió mejorar el presente material. %
%
Finalmente se agradece a los estudiantes que han detectado errores, y a Bruno Bouchard  e Ignacio Suarez por sus contribuciones, logrando mejorar el material destinado a apoyar su propio proceso de aprendizaje.

\begin{flushright}
Jorge Pérez Zerpa\\
\today
\end{flushright} 


\vspace{5mm}

\noindent
\textbf{Sobre el estilo de redacción}

A lo largo del documento se hace uso del género masculino para hacer referencia a una o varias personas ocupando el rol de: lector, estudiante, profesional, etc. La elección de este género busca simplificar la redacción y lectura del material, de todas formas, en todos los casos las referencias incluyen a cualquier persona que ocupe esos roles independientemente del género con el que se identifique.

\vfill
\begin{small}	
	\noindent
	El contenido de este documento es publicado bajo una licencia \textit{Creative Commons Attribution-ShareAlike 4.0 International License}. Ver detalles en creativecommons.org/licenses/by-sa/4.0.\\
	
	\noindent
\textit{
	This work is published under a CC BY-SA license (\textit{Creative Commons Attribution-ShareAlike 4.0 International License}), which means that you can copy, redistribute, remix, transform, and build upon the content for any purpose, even commercially, as long as you give appropriate credit, provide a link to the license, and indicate if changes were made. If you remix, transform, or build upon the material, you must distribute your contributions under the same license as the original. License details: \href{https://creativecommons.org/licenses/by-sa/4.0/}{creativecommons.org/licenses/by-sa/4.0}.
}
	
	\vspace{2mm}
	
	\noindent
\textit{
	The use of general descripitive names, resgistered names, trademarks, etc. in this publication does not imply, even in the absence of a specific statement, that such names are exempt from the relevant protective laws and regulations and therefore free for general use.
}
	
\end{small}

\vspace{3mm}

\noindent
\begin{footnotesize}
	Documento producido usando \textit{software libre}: \href{https://www.latex-project.org/}{\LaTeX}, \href{https://www.texstudio.org/}{TeXstudio}, \href{https://www.geany.org/}{Geany}, \href{https://www.paraview.org/}{Paraview}, \href{https://inkscape.org/}{Inkscape} y \href{https://www.gnu.org/software/octave/}{GNU-Octave}.
\end{footnotesize}
% --------------------------------------------------