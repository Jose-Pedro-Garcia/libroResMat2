\newpage
\begin{table}[H]
	\centering
	\resizebox{0.9\textwidth}{!}
	{
		\begin{tabular}{m{4cm}cc}
			\def\svgscale{0.25}\multicolumn{3}{c}{\Large Propiedades geométricas para torsión} \\ \toprule
			%
			\multicolumn{1}{c}{\large Sección} & \multicolumn{1}{c}{\large J} & \multicolumn{1}{c}{\large $W_t$} \\ \cmidrule{1-3} 
			%
			\def\svgscale{0.45}\input{./figs/TablaTorsion/fig1.pdf_tex} & $\dfrac{\pi D^4}{32}\left(1-\dfrac{d^4}{D^4}\right)$ & $\dfrac{\pi D^3}{16}\left(1-\dfrac{d^4}{D^4}\right)$ \\ \cmidrule{1-3}
			\def\svgscale{0.45}\input{./figs/TablaTorsion/fig2.pdf_tex} & $\dfrac{\pi a^3b^3}{(a^2+b^2)}$ &  $\dfrac{\pi ab^2}{2}$ \\ \cmidrule{1-3}
			\def\svgscale{0.45}\input{./figs/TablaTorsion/fig3.pdf_tex} & $\dfrac{\sqrt{3}}{80}a^4$ & $\dfrac{a^3}{20}$ \\ \cmidrule{1-3}
			\def\svgscale{0.45}\input{./figs/TablaTorsion/fig4.pdf_tex} & $\beta ab^3$ (*) & $\alpha ab^2$ (*) \\ \cmidrule{1-3}
			\def\svgscale{0.45}\input{./figs/TablaTorsion/fig5.pdf_tex} & $\dfrac{st^3}{3}$ & $\dfrac{st^2}{3}$ \\ \cmidrule{1-3}
			\def\svgscale{0.45}\input{./figs/TablaTorsion/fig6.pdf_tex} & $\sum_{i}^{n}\dfrac{s_it_i^3}{3}$ & $\dfrac{J}{t_{m\acute{a}x}}$ \\ \cmidrule{1-3}
			\def\svgscale{0.45}\input{./figs/TablaTorsion/fig7.pdf_tex} & $\dfrac{4\Omega^2}{\int_{s}\dfrac{ds}{t}}$ & $2\Omega t_{min}$ \\ \cmidrule{1-3}
		\end{tabular}
	} % end resizebox
	\caption{$J$ corresponde a la inercia torsional. $W_t$ correspondé al módulo resistente de la sección a torsión. (*): Ver valores de parámetros $\alpha$ y $\beta$ en la \autoref{tab:tabRec}.}
	\label{tab:my_label}
\end{table}


\begin{table}[H]
	\centering
	\resizebox{0.9\textwidth}{!}
	{
	\begin{tabular}{cccccccccccc}
		a/b & 1.00 & 1.50 & 1.75 & 2.00 & 2.50 & 3 & 4 & 6 & 8 & 10 & $\infty$\\ \toprule
		$\alpha$ & 0.208 & 0.231 & 0.239 & 0.246 & 0.258 & 0.267 & 0.282 & 0.299 & 0.307 & 0.313 & 0.333 \\ \cmidrule{1-12}
		$\beta$ & 0.141 & 0.196 & 0.214 & 0.229 & 0.249 & 0.263 & 0.281 & 0.299 & 0.307 & 0.313 & 0.333 \\ \cmidrule{1-12}
	\end{tabular}
	} % end resizebox
	\caption{Parámetros $\alpha$ y $\beta$ para el cálculo de propiedades seccionales de torsión de una sección rectangular.}
	\label{tab:tabRec}
\end{table}			
