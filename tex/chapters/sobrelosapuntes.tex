\subsection*{Sobre este material}

\thispagestyle{empty}

Este documento integra los materiales de apoyo para \textit{asistir} al estudiante en su proceso de aprendizaje de los contenidos de la Unidad Curricular \textit{Resistencia de Materiales 2} de la Carrera Ingeniería Civil brindada por la Facultad de Ingeniería de la Universidad de la República. %
%
El material no está editado y se encuentra en constante proceso de corrección y cambio acompasando las actualizaciones del curso. Se invita a estudiantes o lectores a colaborar publicando errores o realizando modificaciones a través del repositorio abierto \href{https://github.com/jorgepz/libroResMat2}{github.com/jorgepz/libroResMat2}. %
%
Este documento recibió contribuciones de los docentes que han formado parte de la Unidad Curricular, así como también de estudiantes que han encontrado errores. %
Una lista de las contribuciones está disponible en \href{https://github.com/jorgepz/libroResMat2/blob/master/contributors.md}{el archivo \textit{contributors.md}} del repositorio. %
%
Se agradece el apoyo financiero de la Comisión Sectorial de Enseñanza de la Universidad de la República, cuyo financiamiento del proyecto de innovaciones educativas titulado \textit{Rediseño de prácticas de enseñanza y evaluación en Resistencia de Materiales} permitió mejorar el presente material, durante los años 2018 y 2019. %
%
\begin{flushright}
Jorge Pérez Zerpa\\
\today
\end{flushright} 

\vspace{4mm}

\begin{flushleft}
Instituto de Estructuras y Transporte\\
Facultad de Ingeniería, Universidad de la República\\
2022, Montevideo, Uruguay
\end{flushleft} 


\vspace{5mm}


\noindent
\textbf{Sobre el estilo de redacción}

A lo largo del documento se hace uso del género masculino para hacer referencia a una o varias personas ocupando el rol de: lector, estudiante, profesional, etc. La elección de este género busca simplificar la redacción y lectura del material, de todas formas, en todos los casos las referencias incluyen a cualquier persona que ocupe esos roles independientemente del género con el que se identifique.

\vfill
\begin{small}	
	\noindent
	El contenido de este documento es publicado bajo una licencia \textit{Creative Commons Attribution-ShareAlike 4.0 International License}. Ver detalles en
	\href{https://creativecommons.org/licenses/by-sa/4.0/}{creativecommons.org/licenses/by-sa/4.0}.\\
	
	\noindent
\textit{
	This work is published under a CC BY-SA license (\textit{Creative Commons Attribution-ShareAlike 4.0 International License}), which means that you can copy, redistribute, remix, transform, and build upon the content for any purpose, even commercially, as long as you give appropriate credit, provide a link to the license, and indicate if changes were made. If you remix, transform, or build upon the material, you must distribute your contributions under the same license as the original. License details: \href{https://creativecommons.org/licenses/by-sa/4.0/}{creativecommons.org/licenses/by-sa/4.0}.
}
	
	\vspace{2mm}
	
	\noindent
\textit{
	The use of general descripitive names, resgistered names, trademarks, etc. in this publication does not imply, even in the absence of a specific statement, that such names are exempt from the relevant protective laws and regulations and therefore free for general use.
}
	
\end{small}

\vspace{3mm}

\noindent
\begin{footnotesize}
	Documento producido usando \textit{software libre}: \href{https://www.latex-project.org/}{\LaTeX}, \href{https://www.texstudio.org/}{TeXstudio}, \href{https://www.geany.org/}{Geany}, \href{https://www.paraview.org/}{Paraview}, \href{https://inkscape.org/}{Inkscape} y \href{https://www.gnu.org/software/octave/}{GNU-Octave}.
\end{footnotesize}
% --------------------------------------------------